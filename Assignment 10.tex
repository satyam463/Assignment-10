\documentclass[journal,12pt,twocolumn]{IEEEtran}

\usepackage{setspace}
\usepackage{gensymb}

\singlespacing


\usepackage[cmex10]{amsmath}
%\usepackage{amsthm}
%\interdisplaylinepenalty=2500
%\savesymbol{iint}
%\usepackage{txfonts}
%\restoresymbol{TXF}{iint}
%\usepackage{wasysym}
\usepackage{amsthm}
%\usepackage{iithtlc}
\usepackage{mathrsfs}
\usepackage{txfonts}
\usepackage{stfloats}
\usepackage{bm}
\usepackage{cite}
\usepackage{cases}
\usepackage{subfig}
%\usepackage{xtab}
\usepackage{longtable}
\usepackage{multirow}
%\usepackage{algorithm}
%\usepackage{algpseudocode}
\usepackage{enumitem}
\usepackage{mathtools}
\usepackage{graphicx}
\usepackage{refstyle}
\usepackage{caption}
\usepackage{steinmetz}
\usepackage{tikz}
%\usepackage{circuitikz}
\usepackage{verbatim}
\usepackage{tfrupee}
\usepackage[breaklinks=true]{hyperref}
%\usepackage{stmaryrd}
\usepackage{tkz-euclide} % loads  TikZ and tkz-base
%\usetkzobj{all}
\usetikzlibrary{calc,math}
\usepackage{listings}
   \usepackage{color}                                            %%
    \usepackage{array}                                            %%
    \usepackage{longtable}                                        %%
    \usepackage{calc}                                             %%
    \usepackage{multirow}                                         %%
    \usepackage{hhline}                                           %%
    \usepackage{ifthen}                                           %%
  %optionally (for landscape tables embedded in another document): %%
    \usepackage{lscape}     
%\usepackage{multicol}
\usepackage{chngcntr}
%\usepackage{enumerate}

%\usepackage{wasysym}
%\newcounter{MYtempeqncnt}
\DeclareMathOperator*{\Res}{Res}
%\renewcommand{\baselinestretch}{2}
\renewcommand\thesection{\arabic{section}}
\renewcommand\thesubsection{\thesection.\arabic{subsection}}
\renewcommand\thesubsubsection{\thesubsection.\arabic{subsubsection}}

\renewcommand\thesectiondis{\arabic{section}}
\renewcommand\thesubsectiondis{\thesectiondis.\arabic{subsection}}
\renewcommand\thesubsubsectiondis{\thesubsectiondis.\arabic{subsubsection}}

% correct bad hyphenation here
\hyphenation{op-tical net-works semi-conduc-tor}
\def\inputGnumericTable{}                                 %%

\lstset{
%language=C,
frame=single, 
breaklines=true,
columns=fullflexible
}

\begin{document}

\newtheorem{theorem}{Theorem}[section]
\newtheorem{problem}{Problem}
\newtheorem{proposition}{Proposition}[section]
\newtheorem{lemma}{Lemma}[section]
\newtheorem{corollary}[theorem]{Corollary}
\newtheorem{example}{Example}[section]
\newtheorem{definition}[problem]{Definition}

\newcommand{\BEQA}{\begin{eqnarray}}
\newcommand{\EEQA}{\end{eqnarray}}
\newcommand{\define}{\stackrel{\triangle}{=}}
\bibliographystyle{IEEEtran}
%\bibliographystyle{ieeetr}
\providecommand{\mbf}{\mathbf}
\providecommand{\pr}[1]{\ensuremath{\Pr\left(#1\right)}}
\providecommand{\qfunc}[1]{\ensuremath{Q\left(#1\right)}}
\providecommand{\sbrak}[1]{\ensuremath{{}\left[#1\right]}}
\providecommand{\lsbrak}[1]{\ensuremath{{}\left[#1\right.}}
\providecommand{\rsbrak}[1]{\ensuremath{{}\left.#1\right]}}
\providecommand{\brak}[1]{\ensuremath{\left(#1\right)}}
\providecommand{\lbrak}[1]{\ensuremath{\left(#1\right.}}
\providecommand{\rbrak}[1]{\ensuremath{\left.#1\right)}}
\providecommand{\cbrak}[1]{\ensuremath{\left\{#1\right\}}}
\providecommand{\lcbrak}[1]{\ensuremath{\left\{#1\right.}}
\providecommand{\rcbrak}[1]{\ensuremath{\left.#1\right\}}}
\theoremstyle{remark}
\newtheorem{rem}{Remark}
\newcommand{\sgn}{\mathop{\mathrm{sgn}}}
%\providecommand{\abs}[1]{\left\vert#1\right\vert}
\providecommand{\res}[1]{\Res\displaylimits_{#1}} 
%\providecommand{\norm}[1]{\left\lVert#1\right\rVert}
\providecommand{\norm}[1]{\lVert#1\rVert}
\providecommand{\mtx}[1]{\mathbf{#1}}
%\providecommand{\mean}[1]{E\left[ #1 \right]}
\providecommand{\fourier}{\overset{\mathcal{F}}{ \rightleftharpoons}}
%\providecommand{\hilbert}{\overset{\mathcal{H}}{ \rightleftharpoons}}
\providecommand{\system}{\overset{\mathcal{H}}{ \longleftrightarrow}}
	%\newcommand{\solution}[2]{\textbf{Solution:}{#1}}
\newcommand{\solution}{\noindent \textbf{Solution: }}
\newcommand{\cosec}{\,\text{cosec}\,}
\providecommand{\dec}[2]{\ensuremath{\overset{#1}{\underset{#2}{\gtrless}}}}
\newcommand{\myvec}[1]{\ensuremath{\begin{pmatrix}#1\end{pmatrix}}}
\newcommand{\mydet}[1]{\ensuremath{\begin{vmatrix}#1\end{vmatrix}}}
%\numberwithin{equation}{section}
\numberwithin{equation}{subsection}
%\numberwithin{problem}{section}
%\numberwithin{definition}{section}
\makeatletter
\@addtoreset{figure}{problem}
\makeatother
\let\StandardTheFigure\thefigure
\let\vec\mathbf
%\renewcommand{\thefigure}{\theproblem.\arabic{figure}}
\renewcommand{\thefigure}{\theproblem}
%\setlist[enumerate,1]{before=\renewcommand\theequation{\theenumi.\arabic{equation}}
%\counterwithin{equation}{enumi}
%\renewcommand{\theequation}{\arabic{subsection}.\arabic{equation}}
\def\putbox#1#2#3{\makebox[0in][l]{\makebox[#1][l]{}\raisebox{\baselineskip}[0in][0in]{\raisebox{#2}[0in][0in]{#3}}}}
     \def\rightbox#1{\makebox[0in][r]{#1}}
     \def\centbox#1{\makebox[0in]{#1}}
     \def\topbox#1{\raisebox{-\baselineskip}[0in][0in]{#1}}
     \def\midbox#1{\raisebox{-0.5\baselineskip}[0in][0in]{#1}}
\vspace{3cm}
\title{Assignment-10}
\author{Satyam Singh \\ EE20MTECH14015}
\maketitle
\newpage
\bigskip
\renewcommand{\thefigure}{\theenumi}
\renewcommand{\thetable}{\theenumi}
\begin{abstract}
This assignment deals with vector spaces.
\end{abstract}
Download  tex file from 
\begin{lstlisting}
https://github.com/satyam463/Assignment-10/blob/main/Assignment%2010.tex
\end{lstlisting}
\section{Problem Statement}
Let V be the set of all pairs (x,y) of real numbers and let F be the field of real numbers. Define 
\begin{align}
(x,y)+(x_1,y_1)=(x+x_1,y+y_1)\\
c(x,y)=(cx,y)
\end{align}
Is V with these operations , a vector space over the field of real numbers ?
\section{Solution}
V=\{(x,y) $\vert x,y \in R$\} , consider $$u = (x_1,y_1) , v = (x_2,y_2) , w = (x_3,y_3) \in V , a,b,c \in R$$ Axioms with respect to addition and scalar multiplication.
\begin{enumerate}
 \item 
 \begin{align}
 u+v = (x_1,y_1)+(x_2,y_2)\\
     = (x_1+x_2,y_1+y_2)\\
     = (x_2+x_1,y_2+y_1)\\
     = (x_2,y_2)+(x_1,y_1)
     = v + u 
 \end{align}
 \item 
\begin{align}
u+(v+w)\\=(x_1,y_1)+((x_2,y_2)+(x_3,y_3))\\
=(x_1,y_1)+((x_2+x_3),(y_2+y_3))\\
=(x_1+(x_2+x_3),y_1+(y_2+y_3))\\
=((x_1+x_2)+x_3,(y_1+y_2)+y_3)\\
=(x_1+x_2,y_1+y_2)+(x_3,y_3)\\
=(u+v)+w 
 \end{align}
 \item 
 \begin{align}
 u+\vec{0} = (x_1+y_1)+(0,0)\\
     = (x_1+0,y_1+0)\\
     =(x_1,y_1)=u
 \end{align}
 \item 
 \begin{align}
 u + (-u) = (x_1,y_1)+(-x_1,-y_1)\\
          = (x_1+(-x_1),y_1+(-y_1))\\
          = (0 , 0) = \vec{0} 
 \end{align}
 \item 
 \begin{align}
 1.u = 1.(x_1,y_1) = (1.x_1,y_1) = u 
 \end{align}
 \item 
 \begin{align}
  (ab).u = ab.(x_1,y_1)=((ab)x_1,y_1)\\
  =(a(bx_1),y_1)=a(bx_1,y_1)\\
  =ab(x_1,y_1)=a(b.u)
 \end{align}
 \item 
 \begin{align}
 c.(u+v)=c.((x_1,y_1)+(x_2,y_2))\\
 =c.((x_1+x_2),(y_1+y_2))\\
 =(c(x_1+x_2),(y_1+y_2))\\
 =(cx_1+cx_2,y_1+y_2)\\
 =(cx_1,y_1)+(cx_2+y_2))\\
 =c.(x_1,y_1)+c.(x_2+y_2))\\
 =c.u+c.v
 \end{align}
 \item
 \begin{align}
 (a+b).u=(a+b).(x_1,y_1)\\
 =((a+b)x_1,y_1)\neq a.u+b.u\label{contradict}
 \end{align}
\end{enumerate}
Since V with the given operations the equation \refeq{contradict}  contradicts the axioms of scalar multiplication . Hence it is not vector space over real number with these operations.
\end{document}

